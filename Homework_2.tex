% Options for packages loaded elsewhere
\PassOptionsToPackage{unicode}{hyperref}
\PassOptionsToPackage{hyphens}{url}
%
\documentclass[
]{article}
\usepackage{amsmath,amssymb}
\usepackage{iftex}
\ifPDFTeX
  \usepackage[T1]{fontenc}
  \usepackage[utf8]{inputenc}
  \usepackage{textcomp} % provide euro and other symbols
\else % if luatex or xetex
  \usepackage{unicode-math} % this also loads fontspec
  \defaultfontfeatures{Scale=MatchLowercase}
  \defaultfontfeatures[\rmfamily]{Ligatures=TeX,Scale=1}
\fi
\usepackage{lmodern}
\ifPDFTeX\else
  % xetex/luatex font selection
\fi
% Use upquote if available, for straight quotes in verbatim environments
\IfFileExists{upquote.sty}{\usepackage{upquote}}{}
\IfFileExists{microtype.sty}{% use microtype if available
  \usepackage[]{microtype}
  \UseMicrotypeSet[protrusion]{basicmath} % disable protrusion for tt fonts
}{}
\makeatletter
\@ifundefined{KOMAClassName}{% if non-KOMA class
  \IfFileExists{parskip.sty}{%
    \usepackage{parskip}
  }{% else
    \setlength{\parindent}{0pt}
    \setlength{\parskip}{6pt plus 2pt minus 1pt}}
}{% if KOMA class
  \KOMAoptions{parskip=half}}
\makeatother
\usepackage{xcolor}
\usepackage[margin=1in]{geometry}
\usepackage{graphicx}
\makeatletter
\def\maxwidth{\ifdim\Gin@nat@width>\linewidth\linewidth\else\Gin@nat@width\fi}
\def\maxheight{\ifdim\Gin@nat@height>\textheight\textheight\else\Gin@nat@height\fi}
\makeatother
% Scale images if necessary, so that they will not overflow the page
% margins by default, and it is still possible to overwrite the defaults
% using explicit options in \includegraphics[width, height, ...]{}
\setkeys{Gin}{width=\maxwidth,height=\maxheight,keepaspectratio}
% Set default figure placement to htbp
\makeatletter
\def\fps@figure{htbp}
\makeatother
\setlength{\emergencystretch}{3em} % prevent overfull lines
\providecommand{\tightlist}{%
  \setlength{\itemsep}{0pt}\setlength{\parskip}{0pt}}
\setcounter{secnumdepth}{-\maxdimen} % remove section numbering
\usepackage{booktabs}
\usepackage{longtable}
\usepackage{array}
\usepackage{multirow}
\usepackage{wrapfig}
\usepackage{float}
\usepackage{colortbl}
\usepackage{pdflscape}
\usepackage{tabu}
\usepackage{threeparttable}
\usepackage{threeparttablex}
\usepackage[normalem]{ulem}
\usepackage{makecell}
\usepackage{xcolor}
\ifLuaTeX
  \usepackage{selnolig}  % disable illegal ligatures
\fi
\usepackage{bookmark}
\IfFileExists{xurl.sty}{\usepackage{xurl}}{} % add URL line breaks if available
\urlstyle{same}
\hypersetup{
  pdftitle={Homework2.rmd},
  pdfauthor={Fabian \& Samuel},
  hidelinks,
  pdfcreator={LaTeX via pandoc}}

\title{Homework2.rmd}
\author{Fabian \& Samuel}
\date{2024-12-09}

\begin{document}
\maketitle

\subsection{Task 1}\label{task-1}

\begin{verbatim}
## # A tibble: 9 x 2
##   month cold_days
##   <int>     <int>
## 1     1        32
## 2     2        29
## 3     3        32
## 4     4        22
## 5     5        12
## 6     9         3
## 7    10        12
## 8    11        28
## 9    12        28
\end{verbatim}

\begin{verbatim}
## # A tibble: 1 x 1
##   total_missing
##           <int>
## 1          8255
\end{verbatim}

\begin{verbatim}
## # A tibble: 1 x 20
##   total_missing_dep_time  year month   day dep_time sched_dep_time dep_delay
##                    <int> <int> <int> <int>    <int>          <int>     <int>
## 1                      0     0     0     0      521              0       521
## # i 13 more variables: arr_time <int>, sched_arr_time <int>, arr_delay <int>,
## #   carrier <int>, flight <int>, tailnum <int>, origin <int>, dest <int>,
## #   air_time <int>, distance <int>, hour <int>, minute <int>, time_hour <int>
\end{verbatim}

\subsubsection{Explanation}\label{explanation}

The issue of counting more than 31 days in a month arises because the
dataset contains one entry per hour, resulting in 24 rows for each day.
To address this, we grouped the data by date to ensure that each day is
counted only once. The revised approach calculates the minimum
temperature for each day and then filters days where this minimum is
below 10°C. This ensures an accurate count of cold days per month.

Yes, the months with the most cold days (January, February, and
December) align with expectations, as these are winter months in the
Northern Hemisphere. However, it is interesting to note some colder days
in spring (April) and fall (November), which could be attributed to
occasional cold fronts. Unexpectedly cold days in September might
require further investigation for anomalies in the data.

Despite these adjustments, the results still show inconsistencies, such
as months with more than 31 cold days. These discrepancies may arise
from duplicates or irregularities in the weather dataset, such as
multiple recordings for the same hour or incomplete data cleaning.
Without deeper investigation or cleaning of the dataset, it is
impossible to guarantee correct results.

Final Note: The current count of cold days per month should be
interpreted with caution. Additional clarification from the data
provider or further exploration of the dataset might be necessary to
resolve these issues.

\subsection{Task 2}\label{task-2}

\begin{verbatim}
## # A tibble: 19 x 2
##    variable               missing_count
##    <chr>                          <int>
##  1 missing_year                       0
##  2 missing_month                      0
##  3 missing_day                        0
##  4 missing_dep_time                8255
##  5 missing_sched_dep_time             0
##  6 missing_dep_delay               8255
##  7 missing_arr_time                8255
##  8 missing_sched_arr_time             0
##  9 missing_arr_delay               8255
## 10 missing_carrier                    0
## 11 missing_flight                     0
## 12 missing_tailnum                 2512
## 13 missing_origin                     0
## 14 missing_dest                       0
## 15 missing_air_time                8255
## 16 missing_distance                   0
## 17 missing_hour                       0
## 18 missing_minute                     0
## 19 missing_time_hour                  0
\end{verbatim}

\subsubsection{Explanation}\label{explanation-1}

Rows with missing dep\_time likely represent flights that were canceled.
These rows have missing values for other related columns, such as
arr\_time, air\_time, and sched\_dep\_time, because a canceled flight
does not have actual departure or arrival times recorded. Additionally,
the absence of sched\_dep\_time might indicate flights removed from the
schedule altogether due to external factors such as weather conditions,
operational issues, or low demand.

\subsection{Task 3 - a}\label{task-3---a}

\includegraphics{Homework_2_files/figure-latex/unnamed-chunk-3-1.pdf}

\subsection{Task 3 - b}\label{task-3---b}

\begin{verbatim}
## Picking joint bandwidth of 1.58
\end{verbatim}

\includegraphics{Homework_2_files/figure-latex/unnamed-chunk-4-1.pdf}

\subsection{Task 3 - c}\label{task-3---c}

\includegraphics{Homework_2_files/figure-latex/unnamed-chunk-5-1.pdf}

\subsection{Task 4}\label{task-4}

\subsubsection{Data Cleaning and Inspection for
Homework}\label{data-cleaning-and-inspection-for-homework}

\begin{verbatim}
## Warning: There were 3 warnings in `mutate()`.
## The first warning was:
## i In argument: `date of birth = case_when(...)`.
## Caused by warning:
## ! All formats failed to parse. No formats found.
## i Run ]8;;ide:run:dplyr::last_dplyr_warnings()dplyr::last_dplyr_warnings()]8;; to see the 2 remaining warnings.
\end{verbatim}

\begin{verbatim}
## Warning: There were 2 warnings in `mutate()`.
## The first warning was:
## i In argument: `height = case_when(...)`.
## Caused by warning:
## ! NAs durch Umwandlung erzeugt
## i Run ]8;;ide:run:dplyr::last_dplyr_warnings()dplyr::last_dplyr_warnings()]8;; to see the 1 remaining warning.
\end{verbatim}

\begin{verbatim}
## Warning: There was 1 warning in `mutate()`.
## i In argument: `hair = ifelse(hair == "Glatze", 0, as.numeric(hair))`.
## Caused by warning in `ifelse()`:
## ! NAs durch Umwandlung erzeugt
\end{verbatim}

\subsection{Task 5 - A}\label{task-5---a}

\subsection{Task 5 - B}\label{task-5---b}

\begin{verbatim}
## `geom_smooth()` using formula = 'y ~ x'
\end{verbatim}

\includegraphics{Homework_2_files/figure-latex/unnamed-chunk-8-1.pdf}

\subsection{Task 5 - C}\label{task-5---c}

\begin{longtable}[t]{cccccc}
\toprule
Sample & Mean X & Mean Y & SD X & SD Y & Correlation XY\\
\midrule
1 & 9 & 7.501 & 3.317 & 2.032 & 0.816\\
2 & 9 & 7.501 & 3.317 & 2.032 & 0.816\\
3 & 9 & 7.500 & 3.317 & 2.030 & 0.816\\
4 & 9 & 7.501 & 3.317 & 2.031 & 0.817\\
\bottomrule
\end{longtable}

\begin{longtable}[t]{cccccc}
\toprule
Sample & Mean X & Mean Y & SD X & SD Y & Correlation XY\\
\midrule
1 & 9 & 7.501 & 3.317 & 2.032 & 0.816\\
2 & 9 & 7.501 & 3.317 & 2.032 & 0.816\\
3 & 9 & 7.500 & 3.317 & 2.030 & 0.816\\
4 & 9 & 7.501 & 3.317 & 2.031 & 0.817\\
\bottomrule
\end{longtable}

\begin{verbatim}
## `geom_smooth()` using formula = 'y ~ x'
\end{verbatim}

\includegraphics{Homework_2_files/figure-latex/unnamed-chunk-11-1.pdf}

\subsection{\texorpdfstring{\textbf{Workload}}{Workload}}\label{workload}

We sat down at the beginning and divided up the tasks. The first thing
we did was set up a Git repository so that we could easily collaborate
and continuously track each other's progress. We organized the tasks so
that Samuel took on tasks 1, 2 and 3, while Fabian handled tasks 4 and
5. This clear division helped us focus and avoid overlapping efforts.
After completing our initial assignments, we held a short meeting to
discuss the status and share updates on our progress. This check-in
proved beneficial, as it highlighted some areas that needed further
refinement. Although a few items were still incomplete, we felt that we
were moving in the right direction and understood what remained to be
done. To ensure the quality of each other's work, we decided that Fabian
would review and make corrections to Samuel's tasks, and vice versa.
This mutual review process not only helped catch errors but also
facilitated a better understanding of each other's approach. By the end
of the session, we felt more confident about our progress and looked
forward to wrapping up the remaining tasks.

\end{document}
